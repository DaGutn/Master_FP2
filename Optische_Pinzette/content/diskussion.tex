\newpage
\section{Diskussion}
\label{sec:discussion}
Dieser Versuch war in dem Sinne erfolgreich, ein allgemeines Gefühl für die Maßstäbe und für die bei der Optischen Pinzette involvierten Kräfte zu vermitteln.

Die gelegentlich stark abweichenden Werte bei der Kalibrierung der Fallensteifigkeit liegen wahrscheinlich daran, dass die Kalibrierungsdauer mit $\Delta t_{\mathrm{cal}} = \qty{1}{s}$ zu niedrig eingestellt war und die Quarzkugel in ihrer Brown'schen Bewegung durch andere äußere Einflüsse gestört worden ist.

Die berechneten Boltzmannkonstanten sollten dem Überprüfen der Fallensteifigkeiten dienen und so in ungefähr der gleichen Größenordnung wie der Theoriewert liegen.
Bei der Fallensteifigkeit sind die meisten Werte in der richtigen Größenordnung und stimmen mit den Werten anderer Praktikumsgruppen grob überein.
Nach der Umrechnung in die Boltzmannkonstante sind jedoch sehr große Abweichungen zu erkennen.
Das könnte heißen, dass die Varianz der gemessenen Daten zu groß ist.

Die Stokes'sche Fallenkraft konnte aufgrund der technischen Gegebenheiten nicht bestimmt werden, denn dazu wäre das Messen der Flucht-Frequenz $f_{\mathrm{escape}}$ einer Quarzkugel erforderlich gewesen.
Die Flucht-Frequenz ist die maximale Frequenz der sinusförmigen Oszillation der Stage, bis zu der die Kugel noch von der Optischen Pinzette gehalten werden kann.
Das Problem lag daran, dass die Amplitude $A$ dieser Schwingung mit steigender Frequenz kleiner wurde und eventuell kleiner als der Durchmesser der Falle war.
Aus diesem Grund konnte die $f_{\mathrm{escape}}$ nicht mehr mit dem bloßen Auge erkannt werden, da sich die Quarzkugel immer in der Falle befand.

Die Charakterisierung der Vesikelbewegung mit Hilfe der Aktin-Myosin-Motoren in einer Zwiebelzelle ist erfolgreich verlaufen und haben zu unserem Verständnis des Vesikeltransports beigetragen.
Die Messungen der Bremskraft der Myosin-Motoren wurden dadurch erschwert, dass seit Anfang der Messungen ca. 45 Minuten vergangen sind.
Das hatte zur Folge, dass die Aktin Fasen nicht mehr leicht zu erkennen waren und die Bereiche, in denen die Vesikel transportiert wurden sich verbreitert haben.
So wurde es zunehmend schwieriger den Weg der Vesikel zum Abfangen vorauszuahnen.
Außerdem waren die Faser oft nicht parallel zur Probenoberfläche, sondern gingen auch in die Tiefe, sodass man lange nach geeigneten Faserbereichen suchen musste.
