\newpage
\section{Diskussion}
\label{sec:conclusion}

Wie zu erwarten ist bei der Messung ohne Strain-Gauge-Nachregelung \ref{fig:A1}
eine deutliche Verzerrung der Struktur erkennbar.
Die Verzerrung ist auch mit Nachregelung \ref{fig:A2} nicht vollständig verschwunden,
wie an einer $4.7\,\si{\percent}$ Abweichung der Kreisdurchmessergröße in x-Richtung gegenüber der y-Richtung zu erkennen ist.
Insgesammt ist die Struktur mit $2.0\,\si{\percent}$ Abweichung etwas kleiner als in der Anleitung angegeben.

Bei der Quadratstruktur weicht die gemessene gemittelte Größe in x-Richtung um $3.3\,\si{\percent}$ von der y-Richtung ab.
Insgesammt ist auch hier die Struktur mit $0.8\,\si{\percent}$ Abweichung minimal kleiner als angegeben.

Die Streifenstruktur ist $1.2\,\si{\percent}$ größer gemessen als angegeben.


Bei den Messungen der CD, DVD und Blu-Ray fällt auf, dass die Pittiefe viel geringer gemessen wurde 
als in der Literatur angegeben \ref{tab:CD1}. Wahrscheinlich ist, dass hier die Höhenkalibrierung nicht korrekt funktioniert hat.
Auch die maximale Pitlänge hat recht große Abweichungen im Vergleich zur Literatur. Dies lässt sich jedoch wahrscheinlich damit begründen,
dass in dem jeweils kleinen vermessenen Ausschnitt kein maximal langes Pit vorhanden war.
Die anderen Abweichungen sind im Vergleich kleiner und können wahrscheinlich dadurch erklärt werden, dass die Ebenenkorrektur nicht so gut funktioniert hat wie bei der Mikrostruktur.
In Bild \ref{fig:CD_A} ist trotz Korrektur noch ein eindeutiger Untergund zu erkennen. Außerdem wurde die Bestimmung der Maße bei der
Blu-Ray wegen flacheren Pitabfällen ungenauer.

Die ermittelte Speicherkapazität von $(1.77\pm0.05)\,$GB weicht um $164.6\,\si{\percent}$ vom Literaturwert $650\,$GB ab \cite{CD}.
Bei der Berechnung hier wurde jedoch nicht der Lead-In, sowie der Lead-Out Bereich einer CD
beachtet, die die Speicherkapazität verringern. Außerdem sind weit mehr Bits zur Speicherung notwendig, unter
anderem zur Fehlerkorrektur und Synchronisation \cite{CD2}.Es ist jedoch davon auszugehen, dass 
die Abweichung noch immer zu groß ist 



Die Kraft-Abstandskurven weisen alle erwarteten Merkmale auf.
Da bei den Berechnungen jeweils nur die auftretenden Größenordnungen aufgezeigt werden sollten, 
ist kein exakter Vergleich mit der Literatur notwendig.
Die Adhäsionskräfte liegen in der Größenordnung einiger $10\,\si{\nano\newton}$, die Eindringtiefe in der Größenordnung weniger $100\,\si{\nano\meter}$ und das Elastizitätsmodul von Teflon in der Größenordnung von wenigen $10\,\si{\mega\pascal}$.