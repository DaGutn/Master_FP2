\newpage
\section{Diskussion}
\label{sec:conclusion}

Für den $\lambda=405\,\si{\nano\meter}$-Laser konnte Photolumineszenz in allen Proben erzeugt werden.
Dabei wird deutlich, dass sich die Kohlenstoff-Quantenpunkte weniger gut als die Nanokristalle für Photolumineszenz eignen, da das Signal breiter und auch unförmiger (keine Gauß-Form), als das der Nanokristalle ist.
Die bestimmten Kristallgrößen bewegen sich alle in der erwarteten Größenordnung von wenigen Nanometern \cite{anleitung}.

Die Polarisation ist bei allen Proben sehr gering und kann als vernachlässigbar klein angesehen werden.
Dies entspricht der Erwartung, dass die Kristalle unpolarisiertes Licht durch ihre zufällige Orientierung in der Flüssigkeit aussenden.
Außerdem kann bestätigt werden, dass es sich um recht isotrope Kugelförmige Kristalle handelt, da Nanoröhrchen im Gegensatz der Erwartung nach Polarisiertes Licht 
aussenden würden \cite{Rods}.

Die Photolumineszenzintensität zeigt eine lineare Abhängigkeit von der Laserleistung,
wobei eine Sättigung für hohe Intensitäten zu erwarten wäre.
Diese Sättigung stellt sich nicht ein, was darauf schließen lässt, dass die Messung bis $P=17\,$mW nicht ausreichend war.

Die Beobachtung, dass die Photolumineszenz bei höheren Laseranregungswellenlängen 
verschwindet, entspricht der Erwartung, dass die Energie nicht mehr ausreicht um ein Elektron aus dem Valenzband ins Leitungsband anzuregen.
Dabei steht die Intensität der Photolumineszenz nicht im Zusammenhang mit der Anregungswellenlänge, die Unterschiede hier 
sind wahrscheinlich auf die einzelnen Intensitäten der Laserdioden zurückzuführen.
