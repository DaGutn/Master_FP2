\newpage
\section{Diskussion}
\label{sec:disskussion}

Es wurde ein Laser mit einer Pulsdauer von, nach technischen Angaben, ca. $90\,\si{\femto\second}$ verwendet.
Die Messung der Autokorrelationsspur des Lasers ohne jegliche Gegenstände im Strahlengang ergibt eine Pulsdauer von $87,6\,\si{\femto\second}$, was eine Abweichung von $12,4\,\%$ zum Theoriewert darstellt.
Ein möglicher Grund für diese Abweichung könnte sein, dass die in \autoref{fig:Original} (rechts) zu sehenden Schultern nicht in die Ausgleichsrechnung miteinbezogen wurden.

Die Messungen, wo die beiden Bandpassfilter in den Strahlengang gestellt wurden bestätigen den antiproportionalen Zusammenhang zwischen der Halbwertsbreite des Spektrums und der Pulsdauer aufgrund der Fouriertransformation.
Die Bandpassfilter verringern die Pulsbreite und erhöhen somit die Pulsdauer d.h. sie verbreitern folglich die gemessene Autokorrelationsspur in der Zeit.
Wie zu erwarten ist die Pulsdauer für den $12\,\si{\nano\meter}$-Bandpassfilter noch größer als die für den $30\,\si{\nano\meter}$-Bandpassfilter.
Transfor

Die Ergebnisse der Messungen mit verschiedenen Medien im Strahlengang decken sich auch mit der Theorie.
Die Form eines Laserpulses, der ein dispersives Medium passiert hängt von der Strecke in dem Medium ab.
Je länger die Strecke ist, desto mehr werden die verschiedenen Frequenzkomponenten des Pulses durch ihre verschiedenen Phasengeschwindigkeiten verschoben und desto breiter wird der Puls.
In \autoref{fig:Original} (links) ist zu erkennen, dass es sich hier um keine exakt monochromatische Laserstrahlung handelt.
Das heißt also, dass die Pulse eine gewisse Dispersion erfahren.
Die Pulsdauern in \autoref{tab:Glas} werden größer mit steigender Dicke des Glasblocks.
Es ist auch sinvoll, dass die Pulsdauer nach dem Durchgang durch den $12\,\si{\milli\meter}$-Block Si viel stärker verbreitert wird, als beim Durchgang durch Glas ähnlicher Dicke, da mit größerem Brechungsindex $n$ auch die Dispersion steigt.

Die Unsicherheiten der Anpassungsparameter und somit der Pulsdauern wurden weggelassen, da die Bereiche auf denen die Fits beruhen für jede Messung anders gewählt worden sind und die daraus entstandenen Unsicherheiten nicht vergleichbar wären.


