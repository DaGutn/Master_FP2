\newpage
\section{Auswertung}
\label{sec:auswertung}
\subsection{Pulsdauer ohne Filter bzw. Medien}
    Als Erstes wurde der reine Laserstrahl betrachtet, ohne dass ein Filter oder ein Medium in den Strahlengang gestellt wurde.
    Dazu wurde zuerst das Spektrum in \autoref{fig:Original} (links) aufgenommen.
    Es ist ein deutliches Peak bei der zentralen Wellenlänge von \qty{1550}{nm} zu erkennen, doch es sind auch ein paar Neben-Peaks bei ca. 1500, 1535 und \qty{1580}{nm} zu sehen.
    \vspace*{-0.3cm}
    \begin{figure}[ht]
        \centering\captionsetup{format=plain}
        \includegraphics{plots/Original.pdf} \vspace*{-0.5cm}
        \caption{Links ist das Spektrum des originallen Laserstrahls dargestellt, wobei die zentrale Wellenlänge durch die grau gestrichelte Linie gekennzeichnet ist. Rechts ist die Autokorrelationsspur (blau) mit einer angepassten Gaußglocke (rot gestrichelt) zu sehen.}
        \label{fig:Original}
    \end{figure}
    \FloatBarrier
    Die blaue Autokorrelationsspur in \autoref{fig:Original} (rechts) wird durch eine Gauß-Funktion der Formel
    \begin{equation}
        I(t) = A \cdot \exp\left(-4\ln(2)\left(\frac{t}{\Delta\tau}\right)^2\right)
    \end{equation}
    gefittet, wobei $\Delta\tau$ die Halbwertsbreite ist.
    Die dabei benutzten Werte sind mit einem breiten blauen durchsichtigen Streifen markiert und die Gaußglocke wird durch eine rot gestrichelte Linie dargestellt.
    Es sind zwei Schultern zu erkennen, weshalb die Gaußglocke an die oberen Werte angepasst wird.
    Aus dem Fit-Parameter der Halbwertsbreite der Autokorrelationsspur ergibt sich die Pulsdauer
    \begin{equation*}
        \Delta \tau_{\mathrm{orig}} \approx \qty{87,6}{fs}
    \end{equation*}
    über die Formel
    \begin{equation}
        \Delta \tau_{\mathrm{Laser}} = \frac{1}{\sqrt{2}} \cdot \Delta \tau_{\mathrm{Autokorrelation}}\;.
    \end{equation}

\subsection{Pulsdauer mit Bandpassfilter}
    Anschließend wurden Messungen mit einem Bandpassfilter im Strahlengang durchgeführt.
    Das Einsetzen des \qty{30}{nm}-Bandpassfilters ergibt ein Spektrum mit nur einem prominenten Peak.
    Außerhalb des Bereiches $\pm\qty{30}{nm}$ um die zentrale Wellenlänge werden die Intensitäten so stark gedämpft, dass diese nicht mehr von der Diode registriert werden.
    Es ist in \autoref{fig:30nm_Bandpass} (links) jedoch noch ein leichter Neben-Peak bei \qty{1535}{nm} zu erkennen.
    Das liegt daran, dass dieser Neben-Peak innerhalb des \qty{30}{nm}-Bereiches liegt und nicht ganz herausgefiltert wird.    
    \vspace*{-0.3cm}
    \begin{figure}[ht]
        \centering\captionsetup{format=plain}
        \includegraphics{plots/30nm_Bandpass.pdf} \vspace*{-0.5cm}
        \caption{Links ist das Spektrum des Laserstrahls bei Durchgang durch einen \qty{30}{nm}-Bandpassfilter dargestellt, wobei die zentrale Wellenlänge durch die grau gestrichelte Linie gekennzeichnet ist. Rechts ist die Autokorrelationsspur (blau) mit einer angepassten Gaußglocke (rot gestrichelt) zu sehen.}
        \label{fig:30nm_Bandpass}
    \end{figure}
    \FloatBarrier
    In \autoref{fig:30nm_Bandpass} (rechts) sind auch Schultern zu sehen.
    Diese werden wieder nicht bei der Anpassung der Gaußfunktion betrachtet.
    In \autoref{fig:12nm_Bandpass} (links) ist nur noch ein scharfer Peak um die zentrale Wellenlänge zu sehen.
    Es ist zu erkennen, dass die Regression der Autokorrelationsspur hier nur auf einem kleinen Bereich an der Spitze des Peaks beruht.
    Generell wird der Bereich der im Fit einbezogenen Werte so gewählt, dass die angepasste Funktion den Peak bis leicht unter die Hälfte möglichst genau überlagert.
    Die Anpassungen ergeben die Pulsdauern:
    \begin{align*}
        \Delta \tau_{\mathrm{bandpass,30}} \approx \qty{172,2}{fs} \\
        \Delta \tau_{\mathrm{bandpass,12}} \approx \qty{285,9}{fs}
    \end{align*}
    \begin{figure}[ht]
        \centering\captionsetup{format=plain}\vspace*{-1cm}
        \includegraphics{plots/12nm_Bandpass.pdf} \vspace*{-0.5cm}
        \caption{Links ist das Spektrum des Laserstrahls bei Durchgang durch einen \qty{12}{nm}-Bandpassfilter dargestellt, wobei die zentrale Wellenlänge durch die grau gestrichelte Linie gekennzeichnet ist. Rechts ist die Autokorrelationsspur (blau) mit einer angepassten Gaußglocke (rot gestrichelt) zu sehen.}
        \label{fig:12nm_Bandpass}
    \end{figure}
    \FloatBarrier

\subsection{Pulsdauer mit dispersiven Medium}
    \begin{figure}[H]
        \centering\captionsetup{format=plain}\vspace*{-0.5cm}
        \includegraphics{plots/Si_12mm.pdf} \vspace*{-0.5cm}
        \caption{Links ist das Spektrum des Laserstrahls bei Durchgang durch einen \qty{12}{mm}-Block Si dargestellt, wobei die zentrale Wellenlänge durch die grau gestrichelte Linie gekennzeichnet ist. Rechts ist die Autokorrelationsspur (blau) mit einer angepassten Gaußglocke (rot gestrichelt) zu sehen.}
        \label{fig:Si_12mm}
    \end{figure}
    \FloatBarrier
    Hierbei wurden verschiedene dispersive Medien in den Strahlengang eingesetzt.
    Als Erstes wurde ein \qty{12}{mm} dicker Block Si benutzt.
    Da die keine Filter verwendet werden, ähnelt das Spektrum in \autoref{fig:Si_12mm} sehr dem in \autoref{fig:Original}.
    Es ist eine deutliche Verbreiterung der Autokorrelationsspur zu erkennen.
    Die zugehörige Pulsdauer ergibt sich zu
    \begin{equation*}
        \Delta \tau_{\mathrm{Si}} \approx \qty{562,6}{fs}
    \end{equation*}
    Um einen möglichen Zusammenhang besser erkennen zu können, werden die Autokorrelationsspuren bei verschiedenen Glasdicken in einem Plot in \autoref{fig:Glas} zusammengefasst.
    Es wird das gleiche Verfahren angewandt, wie bei den anderen Messungen.
    \begin{figure}[t]
        \centering\captionsetup{format=plain}
        \includegraphics{plots/Glas.pdf} \vspace*{-0.5cm}
        \caption{Die Gauß-Fits der Autokorrelationsspuren bei verschiedenen Glasdicken sind hier dargestellt.}
        \label{fig:Glas}
    \end{figure}
    \FloatBarrier

    Die bestimmten Pulsdauern sind in \autoref{tab:Glas} eingetragen und es ist ein klarer Trend zu erkennen.
    Je dicker das Glas ist, desto größer ist die Pulsdauer.
    \begin{table}[h]
        \centering
        \caption{.}
        \label{tab:Glas}
        \begin{tabular}{c c}
        \toprule
        {Glasdicke [mm]} & {Pulsdauer $\Delta \tau$ [fs]}  \\
        \midrule
        \num{2.85}     &   \num{96,9\pm1,5}  \\
        \num{5.31}     &   \num{97,1\pm1,4}  \\
        \num{96,3}     &   \num{97,4\pm1,4}  \\
        \num{13,51}    &   \num{96,9\pm1,4}  \\
        \num{23,85}    &   \num{98,0\pm1,4}  \\
        \num{23,85}    &   \num{98,7\pm1,5}  \\
        \num{29,83}    &   \num{103,0\pm1,6} \\
        \bottomrule
        \end{tabular}
    \end{table}





